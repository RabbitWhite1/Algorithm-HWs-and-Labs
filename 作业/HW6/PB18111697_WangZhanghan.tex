\documentclass[UTF8]{article}
\usepackage{graphicx}
\usepackage{subfigure}
\usepackage{amsmath}
\usepackage{makecell}
\usepackage[utf8]{inputenc}
\usepackage[space]{ctex} %中文包
\usepackage{listings} %放代码
\usepackage{xcolor} %代码着色宏包
\usepackage{CJK} %显示中文宏包
\usepackage{float}
\usepackage{diagbox}
\usepackage{bm}
\usepackage{ulem} 
\usepackage{amssymb}
\usepackage{soul}
\usepackage{color}
\usepackage{geometry}
\usepackage{fancybox} %花里胡哨的盒子
\usepackage{xhfill} %填充包, 可画分割线 https://www.latexstudio.net/archives/8245
\usepackage{multicol} %多栏包
\usepackage{enumerate} %可以方便地自定义枚举标题
\usepackage{multirow} %表格中多行单元格合并
\usepackage{wasysym} %可以使用wasysym里的一堆奇奇怪怪的符号
\usepackage{hyperref} % url
%%%%%%%%%%%%%%%伪代码%%%%%%%%%%%%%%%
\usepackage{amsmath}
\usepackage{algorithm}
\usepackage{algorithmicx}
\usepackage[noend]{algpseudocode}
%%%%%%%%%%%%%%%画图包%%%%%%%%%%%%%%%
\usepackage{tikz}
\usepackage{pgfplots} % http://pgfplots.sourceforge.net/gallery.html
\usetikzlibrary{pgfplots.patchplots} % 拟合支持
\usetikzlibrary{arrows,shapes,automata,petri,positioning,calc} % 状态图支持
\usetikzlibrary{arrows.meta} % 箭头
\usetikzlibrary{shadows} % 阴影支持
\usepackage{forest} % 画树
\usepackage{adjustbox} % 旋转

\geometry{left = 1.5cm, right = 1.5cm, top=1.5cm, bottom=2cm}

\definecolor{mygreen}{rgb}{0,0.6,0}
\definecolor{mygray}{rgb}{0.5,0.5,0.5}
\definecolor{mymauve}{rgb}{0.58,0,0.82}
\lstset{
	backgroundcolor=\color{white}, 
	%\tiny < \scriptsize < \footnotesize < \small < \normalsize < \large < \Large < \LARGE < \huge < \Huge
	basicstyle = \footnotesize,       
	breakatwhitespace = false,        
	breaklines = true,                 
	captionpos = b,                    
	commentstyle = \color{mygreen}\bfseries,
	extendedchars = false,
	frame = shadowbox, 
	framerule=0.5pt,
	keepspaces=true,
	keywordstyle=\color{blue}\bfseries, % keyword style
	language = C++,                     % the language of code
	otherkeywords={string}, 
	numbers=left, 
	numbersep=5pt,
	numberstyle=\tiny\color{mygray},
	rulecolor=\color{black},         
	showspaces=false,  
	showstringspaces=false, 
	showtabs=false,    
	stepnumber=1,         
	stringstyle=\color{mymauve},        % string literal style
	tabsize=4,          
	title=\lstname           
}

%\sum\nolimits_{j=1}^{M}   上下标位于求和符号的水平右端, 
%\sum\limits_{j=1}^{M}   上下标位于求和符号的上下处, 
%\sum_{j=1}^{M}  对上下标位置没有设定, 会随公式所处环境自动调整. 

%%%%%%%%%%%%%画图包%%%%%%%%%%%%%
\usepackage{tikz}
%%%%%%%%%%%%%好看的矩形%%%%%%%%%%%%%
\tikzset{
	rect1/.style = {
		shape = rectangle,% 指定样式
		minimum height=2cm,% 最小高度
		minimum width=4cm,% 最小宽度
		align = center,% 文字居中
		drop shadow,% 阴影
	}
}
%%%%%%%%%%%%%画图背景包%%%%%%%%%%%%%
\usetikzlibrary{backgrounds}

%%%%%%%%%%%%%在tikz中画一个顶点%%%%%%%%%%%%%
%%%%%%%%%%%%%#1:node名称%%%%%%%%%%%%%
%%%%%%%%%%%%%#2:位置%%%%%%%%%%%%%
%%%%%%%%%%%%%#3:标签%%%%%%%%%%%%%
\newcommand{\newVertex}[3]{\node[circle, draw=black, line width=1pt, scale=0.8] (#1) at #2{#3}}
%%%%%%%%%%%%%在tikz中画一条边%%%%%%%%%%%%%
\newcommand{\newEdge}[2]{\draw [black,very thick](#1)--(#2)}
%%%%%%%%%%%%%在tikz中放一个标签%%%%%%%%%%%%%
%%%%%%%%%%%%%#1:名称%%%%%%%%%%%%%
%%%%%%%%%%%%%#2:位置%%%%%%%%%%%%%
%%%%%%%%%%%%%#3:标签内容%%%%%%%%%%%%%
\newcommand{\newLabel}[3]{\node[line width=1pt] (#1) at #2{#3}}

%%%%%%%%%%%%%强制跳过一行%%%%%%%%%%%%%
\newcommand{\jumpLine} {\hspace*{\fill} \\}
%%%%%%%%%%%%%关键点指令,可用itemise替代%%%%%%%%%%%%%
\newcommand{\keypoint}[2]{$\bullet$\textbf{#1}\quad#2\par}
%%%%%%%%%%%%%<T>平均值表示%%%%%%%%%%%%%
\newcommand{\average}[1]{\left\langle #1\right\rangle }
%%%%%%%%%%%%%表格内嵌套表格%%%%%%%%%%%%%
\newcommand{\tabincell}[2]{\begin{tabular}{@{}#1@{}}#2\end{tabular}}
%%%%%%%%%%%%%大黑点item头%%%%%%%%%%%%%
\newcommand{\itemblt}{\item[$\bullet$]}
%%%%%%%%%%%%%大圈item头%%%%%%%%%%%%%
\newcommand{\itemc}{\item[$\circ$]}
%%%%%%%%%%%%%大星星item头%%%%%%%%%%%%%
\newcommand{\itembs}{\item[$\bigstar$]}
%%%%%%%%%%%%%右▷item头%%%%%%%%%%%%%
\newcommand{\itemrhd}{\item[$\rhd$]}
%%%%%%%%%%%%%定义为%%%%%%%%%%%%%
\newcommand{\defas}{=_{df}}
%%%%%%%%%%%%%偏导%%%%%%%%%%%%%
\newcommand{\partialx}[2]{\frac{\partial #1}{\partial #2}}
%%%%%%%%%%%%%蕴含%%%%%%%%%%%%%
\newcommand{\imp}{\rightarrow}
%%%%%%%%%%%%%上取整%%%%%%%%%%%%%
\newcommand{\ceil}[1]{\lceil#1\rceil}
%%%%%%%%%%%%%下取整%%%%%%%%%%%%%
\newcommand{\floor}[1]{\lfloor#1\rfloor}

%%%%%%%%%%%%%双线分割线%%%%%%%%%%%%%
\newcommand*{\doublerule}{\hrule width \hsize height 1pt \kern 0.5mm \hrule width \hsize height 2pt}
%%%%%%%%%%%%%双线中间可加东西的分割线%%%%%%%%%%%%%
\newcommand\doublerulefill{\leavevmode\leaders\vbox{\hrule width .1pt\kern1pt\hrule}\hfill\kern0pt }
%%%%%%%%%%%%%左大括号%%%%%%%%%%%%%
\newcommand{\leftbig}[1]{\left\{\begin{array}{l}#1\end{array}\right.}
%%%%%%%%%%%%%矩阵%%%%%%%%%%%%%
\newcommand{\mat}[2]{\left[\begin{array}{#1}#2\end{array}\right]}
%%%%%%%%%%%%%可换行圆角文本框%%%%%%%%%%%%%
\newcommand{\ovalboxn}[1]{\ovalbox{\tabincell{l}{#1}}}
%%%%%%%%%%%%%设置section的counter, 使从1开始%%%%%%%%%%%%%
\setcounter{section}{0}

%%%%%%%%%%%%%Colors%%%%%%%%%%%%%
\newcommand{\lightercolor}[3]{% Reference Color, Percentage, New Color Name
	\colorlet{#3}{#1!#2!white}
}
\newcommand{\darkercolor}[3]{% Reference Color, Percentage, New Color Name
	\colorlet{#3}{#1!#2!black}
}
\definecolor{aquamarine}{rgb}{0.5, 1.0, 0.83}
\definecolor{Seashell}{RGB}{255, 245, 238} %背景色浅一点的
\definecolor{Firebrick4}{RGB}{255, 0, 0}%文字颜色红一点的
\lightercolor{gray}{20}{lgray}
\newcommand{\hlg}[1]{
	\begingroup
	\sethlcolor{lgray}%背景色
	\textcolor{black}{\hl{\mbox{#1}}}%textcolor里面对应文字颜色
	\endgroup
}
\lightercolor{red}{20}{lred}
\lightercolor{black}{20}{lblack}
\lightercolor{black}{60}{mblack}

\title{算法基础 HW6}
\author{PB18111697 王章瀚}

\begin{document}
\maketitle
\section{}
\noindent \textbf{我们对钢条切割问题进行一点修改, 除了切割下的钢条段具有不同价格$p_i$外, 每次切割还要付出固定的成本$c$. 这样, 切割方案的收益就等于钢条段的价格之和减去切割的成本. 设计一个动态规划算法解决修改后的钢条切割问题.}

\begin{minipage}{\linewidth*6/7}
	\begin{algorithm}[H]
		\caption{考虑切割成本的钢条切割问题}
		\begin{algorithmic}[1] %每行显示行号
			\Require 长度为 $i$ 的钢条价格为 $p_i$, 每次切割有固定成本 $c$, 钢管长度为 $n$
			\Function {\sc Bottom-Up-Cut-Rod}{p, c}
			\State let r[0..n] be a new array
			\State $r[0]=0$
			\For{$j = 1 \to n$}
				\State $q=-\infty$
				\For{$i = 1 \to j$}
					\State $q=\max(q,p[i]+r[j-i]-c)$ \Comment{固定成本 $c$ 在这里被考虑}
				\EndFor
				\State r[j]=q
			\EndFor
			\State \textbf{return} $r[n]$
			\EndFunction
		\end{algorithmic}
	\end{algorithm}
\end{minipage}

\newpage
\section{}
\noindent \textbf{给定一个有向无环图 $G=(V;E)$, 边权重为实数, 给定图中两个顶点 $s$ 和 $t$. 设计动态规划算法, 求从 $s$ 到 $t$ 的最长加权简单路径.}

\noindent 这里我们显然有: 如果某结点 $v$ 是该路径上的, 那么沿着这个路径, $s$ 到 $v$ 和 $v$ 到 $t$ 也分别是最长加权简单路径. 否则一旦有更长的, 换上去就行了. \\
这是因为这个更长$v\rightsquigarrow_{new} t$ (或 $s\rightsquigarrow_{new} v$)的显然不会经过 $s\rightsquigarrow v$ (或$v\rightsquigarrow t$)上的顶点.3 \\
假设经过, 比如是 $s \rightsquigarrow_{new}v$ 经过了 $v\rightsquigarrow t$ 上的顶点$\omega$, 那么就有$\omega\rightsquigarrow_{new} v$ 和 $v\rightsquigarrow \omega$, 这样一来就构成了环, 与题设矛盾. 另一种情况同理.\\
故确实可以利用上述思想做动态规划.\\
\begin{minipage}{\linewidth*6/7}
	\begin{algorithm}[H]
		\caption{有向无环图的最长加权简单路径}
		\begin{algorithmic}[1] %每行显示行号
			\Require 给定有向无环图$G=(V,E)$, 两个顶点$s$, $t$
			\Function {\sc Longest-Weighted-Path}{V, E, s, t}
				\For {$v$ in $V$}
					\State $v.val=-\infty$
					\State $v.last = NULL$
				\EndFor
				\State $s.val = 0$
				\State let $S = \{s\}$
				\While{$S$ is not $\varnothing$}
					\For{$v$ in $S$}
						\State $S = S/\{v\}$
						\For {$w$ in $v.to\_list$}
							\If{$w.val < edge(v,w).weight + v.val$}
								\State $w.val = edge(v,w).weight + v.val$
								\State $w.from = v$
							\EndIf
							\State $S = S\cup \{w\}$
						\EndFor
					\EndFor
				\EndWhile
				\State let $path$ = a list
				\State $v = t$
				\While{$v is not NULL$}
					\State path.insert(v, path.begin) \Comment 把 $v$ 插入列表首部
					\State $v = v.last$
				\EndWhile
				\State \textbf{return} path
				
			\EndFunction
		\end{algorithmic}
	\end{algorithm}
\end{minipage}

\newpage
\section{}
\noindent \textbf{在最优二叉搜索树问题中, 若给定7个关键字的概率如下所示, 求其最优二叉搜索树的结构和代价.(p, q定义和课本相同)}
\begin{center}
	\begin{tabular}{c|c|c|c|c|c|c|c|c}
		i & 0 & 1 & 2 & 3 & 4 & 5 & 6 & 7 \\
		\hline
		$p_i$ &  & 0.04 & 0.06 & 0.08 & 0.02 & 0.10 & 0.12 & 0.14 \\
		$q_i$ & 0.06 & 0.06 & 0.06 & 0.06 & 0.05 & 0.05 & 0.05 & 0.05 \\
	\end{tabular}
\end{center}
\noindent 首先可以画出以下三个表:
\begin{center}
\begin{tabular}{c|c|c|c|c|c|c|c|c|}
	\diagbox{j}{w}{i}&1&2&3&4&5&6&7&8\\
	\cline{0-8}
	7 & 1.0 & 0.90 & 0.78 & 0.64 & 0.56 & 0.41 & 0.24 & 0.05\\
	\cline{0-8}
	6 & 0.81 & 0.71 & 0.59 & 0.45 & 0.37 & 0.22 & 0.05\\
	\cline{0-7}
	5 & 0.64 & 0.54 & 0.42 & 0.28 & 0.20 & 0.05\\
	\cline{0-6}
	4 & 0.49 & 0.39 & 0.27 & 0.13 & 0.05\\
	\cline{0-5}
	3 & 0.42 & 0.32 & 0.20 & 0.06\\
	\cline{0-4}
	2 & 0.28 & 0.18 & 0.06\\
	\cline{0-3}
	1 & 0.16 & 0.06\\
	\cline{0-2}
	0 & 0.06\\
	\cline{0-1}
\end{tabular}
\end{center}
\begin{minipage}{\linewidth*2/3}
\begin{tabular}{c|c|c|c|c|c|c|c|c|}
\diagbox{j}{e}{i}&1&2&3&4&5&6&7&8\\
\cline{0-8}
7 & 3.12 & 2.61 & 2.12 & 1.55 & 1.20 & 0.78 & 0.34 & 0.05\\
\cline{0-8}
6 & 2.44 & 1.96 & 1.48 & 1.01 & 0.72 & 0.32 & 0.05\\
\cline{0-7}
5 & 1.83 & 1.41 & 1.04 & 0.57 & 0.30 & 0.05\\
\cline{0-6}
4 & 1.34 & 0.93 & 0.57 & 0.24 & 0.05\\
\cline{0-5}
3 & 1.02 & 0.68 & 0.32 & 0.06\\
\cline{0-4}
2 & 0.62 & 0.30 & 0.06\\
\cline{0-3}
1 & 0.28 & 0.06\\
\cline{0-2}
0 & 0.06\\
\cline{0-1}
\end{tabular}
\end{minipage}
\begin{minipage}{\linewidth/3}
\begin{tabular}{c|c|c|c|c|c|c|c|}
	\diagbox{j}{root}{i}&1&2&3&4&5&6&7\\
	\cline{0-7}
	7 & 5 & 5 & 5 & 6 & 6 & 7 & 7\\
	\cline{0-7}
	6 & 3 & 5 & 5 & 5 & 6 & 6\\
	\cline{0-6}
	5 & 3 & 3 & 4 & 5 & 5\\
	\cline{0-5}
	4 & 2 & 3 & 3 & 4\\
	\cline{0-4}
	3 & 2 & 3 & 3\\
	\cline{0-3}
	2 & 2 & 2\\
	\cline{0-2}
	1 & 1\\
	\cline{0-1}
\end{tabular}
\end{minipage}
\jumpLine
\noindent从而能够得出最优二叉搜索树:
\begin{center}
\begin{forest}
	[5
		[2
			[1
				[$d_0$]
				[$d_1$]
			]
			[3
				[$d_2$]
				[4
					[$d_3$]
					[$d_4$]
				]
			]
		]
		[7
			[6
				[$d_5$]
				[$d_6$]
			]
			[$d_7$]
		]
	];
\end{forest}
\end{center}
其代价即为 $e[1,7]=3.12$

\newpage
\section{}
\noindent \textbf{一位公司主席正在向Stewart教授咨询公司聚会方案. 公司的内部结构关系是层次化的, 即员工按主管-下属关系构成一棵树, 根结点为公司主席. 人事部按"宴会交际能力"为每个员工打分, 分值为实数. 为了使所有参加聚会的员工都感到愉快, 主席不希望员工及其直接主管同时出席. \\
公司主席向Stewart教授提供公司结构树, 采用左孩子右兄弟表示法(参见课本10.4节)描述. 每个节点除了保存指针外, 还保存员工的名字和宴会交际评分. 设计算法, 求宴会交际评分之和最大的宾客名单. 分析算法复杂度. }
\jumpLine
\jumpLine
采用动态规划的思想, 如果最优方案选中了 $x$, 那么其所有孙子(因为孩子不会被选)的选择情况将也是最优的, 否则通过替换即可得到更优解.\\
\noindent因此如果令 $r[x]$ 表示以 $x$ 为根的子树能够构成的最佳宴会名单对应的评分和, 那么就有
$$\mbox{若不邀请x: }r_0[x]=\max\left\{\begin{array}{l}
	\sum\limits_{\mbox{\tiny y in x.children}}r_0[y]\\
	\sum\limits_{\mbox{\tiny y in x.grandchildren}}r_0[y]
\end{array}\right.$$
$$\mbox{若邀请x: }r_1[x]=x.score+\sum\limits_{\mbox{\tiny y in x.grandchildren}}r_0[y]$$

\noindent 因此可以写出相应算法\footnote{参考自\url{https://blog.csdn.net/yangtzhou/article/details/84305563}}:\\
其通过自上而下递归遍历结点, 且若遇到已经算完的情况就能自动返回(动态规划的关键), 并且最后计算出名单只需要遍历一遍, 因此\textbf{复杂度是 $\Theta(n)$}.\\
\textbf{具体算法见下页}\\
\begin{minipage}{\linewidth*7/7}
	\begin{algorithm}[H]
		\caption{宴会邀请}
		\begin{algorithmic}[1] %每行显示行号
			\Require 员工结构树 $T$, 返回显示是否被邀请的表 $guests$
			\Function {\sc Best-Invitation}{$T$} \Comment{此后, 把 $guest$ 中被邀请者提取即为名单}
			\State $r_0$, $r_1$ 为结点到 $r$ 值的表, 其初始值均为 $-\infty$
			\State $guest$ 指示是否被邀请, 初值为 $False$
			\State {\sc Best-Invitation}($T, r_0, r_1, guests$)
			\If{$r_0[T]>r_1[T]$}
				\State $guests[T]=False$
				\State \Return $r_0[T], guests$
			\Else
				\State $guests[T]=True$
				\State \Return $r_1[T], guests$
			\EndIf
			\EndFunction
			\State
			\Function {\sc Best-Invitation-Sub}{$T, \&r_0, \&r_1, \&guests$}
			\If{$T$ is NULL}
				\State \Return 
			\EndIf
			\If{$r_0[T] != \infty$} \Comment{已经算过}
				\State \Return
			\EndIf
			\State children = T.children
			\State $r_0[T]=0$
			\State $r_1[T]=T.score$
			\For{x in children}
				\State {\sc Best-Invitation}($x, r_0, r_1, guests$)
				\If{$r_0[x]>r_1[x]$} \Comment{不邀请 $T$ 的情况下, 邀请了 $x$ 能得到更好的结果}
					\State $r_0[T] += r_0[x]$
				\Else \Comment{不邀请 $T$ 的情况下, 不邀请 $x$ 能得到更好的结果}
					\State $r_0[T] += r_1[x]$
				\EndIf
				\State $r_1[T] += T.score + r_0[x]$  \Comment{邀请 $T$ 的情况下, 只要直接加上不邀请孩子 $x$ 的 $r$ 值即可}
			\EndFor
			\If{$r_0[T]>r_1[T]$}
				\State $guests[T]=False$
			\Else
				\State $guests[T]=True$
			\EndIf
			\EndFunction
		\end{algorithmic}
	\end{algorithm}
\end{minipage}

\newpage 
\section{}
\textbf{设计一个高效的算法, 对实数线上给定的一个点集$\{x_1,x_2,\cdots,x_n\}$, 求一个单位长度闭区间的集合, 包含所有给定的点, 并要求此集合最小. 证明你的算法是正确的.}\\
\jumpLine

\noindent \textbf{首先给出算法}: 
\begin{center}
\begin{enumerate}[1. ]
	\item 将点集排序得到列表 $X$, 初始化区间集合为 $sections = \varnothing$
	\item 取 $X$ 中最小元 $x_i$, 并取区间 $s = [x_i, x_i+1]$\\
	将 $X$ 中落在区间 $s$ 中的点去掉, 并令 $sections = sections\cup s$
	\item 如果 $X\not=\varnothing$, 继续执行第 2 步, 否则进入 第4步
	\item 此时区间集合 $S$ 即为所求集合
\end{enumerate}
\end{center}
\jumpLine
\noindent \textbf{现在证明其正确性}:\\
首先显然如果区间左端点不是点集中的点之一, 那么完全可以使之增大直到达到的第一个点, 这样只可能减小区间集合大小.\\
其次, 在第一次进入第二步的时候, 不论如何一定要有一个区间包含第一个点$x_1$, 那么根据前述, 它必须是$[x_1,x_1+1]$, 此时显然可以不考虑落在该区间的点了. 然后重复地执行, 每次都是全局最优方案之选, 因此整个算法是正确的. 
\hfill $\square$
%首先显然如果区间左端点不是点集中的点之一, 那么完全可以使之增大直到达到的第一个点, 这样只可能减小区间集合大小.\\
%既然如此, 不妨设 $S=\{[x_1,x_1+1],\cdots,[x_n,x_n+1]\}$, $\mathcal{I}=\{s|\mbox{s为S的任意区间互不相交的子集}\}$. \\
%显然
%\begin{enumerate}[1]
%	\item 如果$B\in \mathcal{I}$ 且 $A\subset B$, 由于区间不相交集合的子集显然也区间不相交, 故$A\in \mathcal{I}$
%	\item 对于任意 $A,B\in\mathcal{I}$且$|A|<|B|$, 则存在元素$x\in B-A$, 使得$A\cup \{x\}\in\mathcal{I}$, 因为$B\in \mathcal{I}$ 表明了在$[x_1,x_n+1]$里存在$|B|$个区间互不相交,
%\end{enumerate}
%故$M=(S,\mathcal{I})$ 满足拟阵性质.\\
%而上述算法过程实际上就是拟阵的贪心算法过程, 因为它每次从 $M.S$ 中取了最小元尝试加入.\\
%综上, 上述算法是正确的. 
%\hfill $\square$

\newpage
\section{}
\textbf{考虑用最少的硬币找n美分零钱的问题. 假定每种硬币的面额都是整数. 设计贪心算法求解找零问题, 假定有25美分, 10美分, 5美分和1美分四种面额的硬币. 证明你的算法能找到最优解.}\\
\jumpLine
\textbf{首先给出算法}:
\begin{enumerate}[1. ]
	\item 先选$a=n/25$个25美分硬币
	\item 再选$b=(n-25*a)/10$个10美分硬币
	\item 再选$c=(n-25*a-10*b)/5$个5美分硬币
	\item 再选$(n-25*a-10*b-5*c)=d$ 个1美分硬币
	\item 最终得到结果: a个25美分硬币, b 个10美分硬币, c个5美分硬币, d个1美分硬币
\end{enumerate}
\jumpLine
\textbf{下面证明算法的最优性}:
如果不是最优, 假设 $(a',b',c',d')$ 为一更优解,\\
若$a'<a$, 则$(0,b',c',d')$的价值必然至少比$(0,b,c,d)$的刚好大$25*(a-a')$, 此时可以从中凑出 $(a-a')$个25, 显然这能够找出比$(a',b',c',d')$更小的结果.\\
依此类推, 若$b'<b$, 若$c'<c$, 若$d'<d$均有类似结论.
所以$a'\ge a$, $b'\ge b$, $c'\ge c$, $d'\ge d$, 因此$(a,b,c,d)$必然是最优解.\\
\hfill$\square$

%不妨设 n = 25a + 10b + 5c + d, 其中 a,b,c,d 为按上述算法计算的结果。
%
%若(a, b, c, d) 不为最优解,不妨设最优解为 (a', b', c', d'), 其中 $a \neq a'$. 则必有 $a > a'$, 故有 $25(a-a') = 10(b'-b) + 5(c'-c) + (d'-d) \Rightarrow \triangle_b + \triangle_c + \triangle_d > (a-a')$



\end{document}









