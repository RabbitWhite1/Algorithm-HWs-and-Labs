\documentclass[UTF8]{article}
\usepackage{graphicx}
\usepackage{subfigure}
\usepackage{amsmath}
\usepackage{makecell}
\usepackage[utf8]{inputenc}
\usepackage[space]{ctex} %中文包
\usepackage{listings} %放代码
\usepackage{xcolor} %代码着色宏包
\usepackage{CJK} %显示中文宏包
\usepackage{float}
\usepackage{makecell}
\usepackage{diagbox}
\usepackage{bm}
\usepackage{ulem} 
\usepackage{amssymb}
\usepackage{soul}
\usepackage{color}
\usepackage{geometry}
\usepackage{fancybox} %花里胡哨的盒子
\usepackage{xhfill} %填充包, 可画分割线 https://www.latexstudio.net/archives/8245
\usepackage{multicol} %多栏包
\usepackage{enumerate} %可以方便地自定义枚举标题
\usepackage{multirow} %表格中多行单元格合并
\usepackage{wasysym} %可以使用wasysym里的一堆奇奇怪怪的符号
%%%%%%%%%%%%%%%伪代码%%%%%%%%%%%%%%%
\usepackage{amsmath}
\usepackage{algorithm}
\usepackage[noend]{algpseudocode}
%%%%%%%%%%%%%%%画图包%%%%%%%%%%%%%%%
\usepackage{tikz}
\usepackage{tikZ-timing} % 时序图支持
\usepackage{pgfplots} % http://pgfplots.sourceforge.net/gallery.html
\usetikzlibrary{pgfplots.patchplots} % 拟合支持
\usetikzlibrary{arrows,shapes,automata,petri,positioning,calc} % 状态图支持
\usetikzlibrary{shadows} % 阴影支持
\usepackage{forest} % 画树

\geometry{left = 1.5cm, right = 1.5cm, top=1.5cm, bottom=2cm}

\definecolor{mygreen}{rgb}{0,0.6,0}
\definecolor{mygray}{rgb}{0.5,0.5,0.5}
\definecolor{mymauve}{rgb}{0.58,0,0.82}
\lstset{
	backgroundcolor=\color{white}, 
	%\tiny < \scriptsize < \footnotesize < \small < \normalsize < \large < \Large < \LARGE < \huge < \Huge
	basicstyle = \footnotesize,       
	breakatwhitespace = false,        
	breaklines = true,                 
	captionpos = b,                    
	commentstyle = \color{mygreen}\bfseries,
	extendedchars = false,
	frame = shadowbox, 
	framerule=0.5pt,
	keepspaces=true,
	keywordstyle=\color{blue}\bfseries, % keyword style
	language = C++,                     % the language of code
	otherkeywords={string}, 
	numbers=left, 
	numbersep=5pt,
	numberstyle=\tiny\color{mygray},
	rulecolor=\color{black},         
	showspaces=false,  
	showstringspaces=false, 
	showtabs=false,    
	stepnumber=1,         
	stringstyle=\color{mymauve},        % string literal style
	tabsize=4,          
	title=\lstname           
}

%%%%%%%%%%%%%画图包%%%%%%%%%%%%%
\usepackage{tikz}
%%%%%%%%%%%%%好看的矩形%%%%%%%%%%%%%
\tikzset{
  rect1/.style = {
    shape = rectangle,% 指定样式
    minimum height=2cm,% 最小高度
    minimum width=4cm,% 最小宽度
    align = center,% 文字居中
    drop shadow,% 阴影
  }
}
%%%%%%%%%%%%%画图背景包%%%%%%%%%%%%%
\usetikzlibrary{backgrounds}

%%%%%%%%%%%%%在tikz中画一个顶点%%%%%%%%%%%%%
%%%%%%%%%%%%%#1:node名称%%%%%%%%%%%%%
%%%%%%%%%%%%%#2:位置%%%%%%%%%%%%%
%%%%%%%%%%%%%#3:标签%%%%%%%%%%%%%
\newcommand{\newVertex}[3]{\node[circle, draw=black, line width=1pt, scale=0.8] (#1) at #2{#3}}
%%%%%%%%%%%%%在tikz中画一条边%%%%%%%%%%%%%
\newcommand{\newEdge}[2]{\draw [black,very thick](#1)--(#2)}
%%%%%%%%%%%%%在tikz中放一个标签%%%%%%%%%%%%%
%%%%%%%%%%%%%#1:名称%%%%%%%%%%%%%
%%%%%%%%%%%%%#2:位置%%%%%%%%%%%%%
%%%%%%%%%%%%%#3:标签内容%%%%%%%%%%%%%
\newcommand{\newLabel}[3]{\node[line width=1pt] (#1) at #2{#3}}

%%%%%%%%%%%%%强制跳过一行%%%%%%%%%%%%%
\newcommand{\jumpLine} {\hspace*{\fill} \par}
%%%%%%%%%%%%%关键点指令,可用itemise替代%%%%%%%%%%%%%
\newcommand{\average}[1]{\left\langle #1\right\rangle }
%%%%%%%%%%%%%表格内嵌套表格%%%%%%%%%%%%%
\newcommand{\keypoint}[2]{$\bullet$\textbf{#1}\quad#2\par}
%%%%%%%%%%%%%<T>平均值表示%%%%%%%%%%%%%
\newcommand{\tabincell}[2]{\begin{tabular}{@{}#1@{}}#2\end{tabular}}%放在导言区
%%%%%%%%%%%%%大黑点item头%%%%%%%%%%%%%
\newcommand{\itemblt}{\item[$\bullet$]}
%%%%%%%%%%%%%大圈item头%%%%%%%%%%%%%
\newcommand{\itemc}{\item[$\circ$]}
%%%%%%%%%%%%%大星星item头%%%%%%%%%%%%%
\newcommand{\itembs}{\item[$\bigstar$]}
%%%%%%%%%%%%%右▷item头%%%%%%%%%%%%%
\newcommand{\itemrhd}{\item[$\rhd$]}
%%%%%%%%%%%%%定义为%%%%%%%%%%%%%
\newcommand{\defas}{=_{df}}
%%%%%%%%%%%%%蕴含%%%%%%%%%%%%%
\newcommand{\imp}{\rightarrow}
%%%%%%%%%%%%%上取整%%%%%%%%%%%%%
\newcommand{\ceil}[1]{\lceil#1\rceil}
%%%%%%%%%%%%%下取整%%%%%%%%%%%%%
\newcommand{\floor}[1]{\lfloor#1\rfloor}

%%%%%%%%%%%%%双线分割线%%%%%%%%%%%%%
\newcommand*{\doublerule}{\hrule width \hsize height 1pt \kern 0.5mm \hrule width \hsize height 2pt}
%%%%%%%%%%%%%双线中间可加东西的分割线%%%%%%%%%%%%%
\newcommand\doublerulefill{\leavevmode\leaders\vbox{\hrule width .1pt\kern1pt\hrule}\hfill\kern0pt }
%%%%%%%%%%%%%左大括号%%%%%%%%%%%%%
\newcommand{\leftbig}[1]{\left\{\begin{array}{l}#1\end{array}\right.}
%%%%%%%%%%%%%矩阵%%%%%%%%%%%%%
\newcommand{\mat}[2]{\left[\begin{array}{#1}#2\end{array}\right]}
%%%%%%%%%%%%%可换行圆角文本框%%%%%%%%%%%%%
\newcommand{\ovalboxn}[1]{\ovalbox{\tabincell{l}{#1}}}
%%%%%%%%%%%%%设置section的counter, 使从0开始%%%%%%%%%%%%%
\setcounter{section}{-1}

\title{算法基础 HW1}
\author{PB18111697 王章瀚}

\begin{document}
\maketitle
\begin{enumerate}[1.]
\item 
	\begin{enumerate}[(a).]
	\item 线性查找伪代码如下:\\
		\begin{minipage}{0.5\linewidth}
			\begin{algorithm}[H]
			\caption{线性查找} 
				\begin{algorithmic}[1]
				\Function{lin\_find}{$A$, $v$}
					\State n = len(A)
					\For {$i = 1,2,\cdots n$}
						\If{$a_i == v$}
							\State return i
						\EndIf
					\EndFor
					\State return NIL
				\EndFunction
				\end{algorithmic} 
			\end{algorithm}
		\end{minipage}\\
		\jumpLine
		使用循环不变式证明算法正确性:
		\begin{itemize}
		\item \textbf{初始化}: 每次循环开始前, 必然有$A[1..i-1]\not=v$, 这就是循环不变式
		\item \textbf{保持}: 如果$a_i!=v$, 那么可以推出下次迭代有$A[1..i]\not=v$, 即下次迭代也为真
		\item \textbf{终止}: 如果$a_i==v$,那么就得到了正确的$i$, 直接返回它. 如果循环迭代结束了还没有找到等于$v$的$a_i$那么就会返回$NIL$表示找不到.
		\end{itemize}
	\item 若$v$在$A$中第$i$个, 那就需要查找$i$个元素, 按题述不需要考虑$v$不在$A$的情况. 因此平均需要检查元素个数为:$$\frac{1}{n}\sum\limits_{i=1}^{n}i=\frac{n+1}{2}$$
	最坏情况就是每次都是最后一个元素, 那么需要查找元素个数为: $n$\\
	因此有
	\begin{tabular}{|c|c|}
	\hline
	平均情况 & $\Theta(\frac{n+1}{2})=\Theta(n)$ \\
	\hline
	最坏情况 & $\Theta(n)$ \\
	\hline
	\end{tabular}
	\end{enumerate}
	\jumpLine
\item 假定$f(n)$与$g(n)$都是渐进非负函数, 判断下列等式或陈述是否一定是正确的, 并简要解释你的答案.
	\begin{enumerate}[a ]
	\item $f(n)=O(f(n)^2)$
	\item $f(n)+g(n)=\Theta(\max(f(n),g(n)))$
	\item $f(n)+O(f(n))=\Theta(f(n))$
	\item if $f(n)=\Omega(g(n))$, then $f(n)=o(g(n))$
	\end{enumerate}
	\begin{enumerate}[a.]
	\item \textbf{错误}. 假设$f(n)=O(f(n)^2)$成立,那么对足够大的$n$就有$0\le f(n)\le cf(n)^2$, 从而得到$1\le cf(n)$, 因此只要构造$f(n)$为单调递减的函数(如$\frac{1}{n}$), 则不能对充分大的$n$满足$\frac{1}{c}\le f(n)$.
	\item \textbf{正确}. 记$\phi(n)=\max(f(n),g(n))$ 只需要证明对于充分大的$n$, 存在$c_1,c_2>0$使得$$0\le c_1\phi(n)\le f(n)+g(n)\le c_2\phi(n)$$
	事实上, 由于$\phi(n)=\max(f(n),g(n))$, 那么取$c_2=2$必然有$f(n)+g(n)\le c_2\phi(n)$\\
	另一方面, 因为有渐进非负性, 必然有$\phi(n)\le f(n)+g(n)$, 取$c_1=1$即可.
	\item \textbf{正确}. 对于$O(f(n))$有, 对充分大的$n$存在$0\le O(f(n))\le c_0f(n)$,
	从而得到$$f(n)\le O(f(n))+f(n)=\Theta(n)\le (c_0+1)f(n)$$
	因此$f(n)+O(f(n))=\Theta(f(n))$
	\item \textbf{错误}. 不妨取$f(n)=g(n)=n^2$, 那么显然$f(n)=\Omega(g(n))$, 但$\lim\limits_{n\rightarrow\infty}\frac{f(n)}{g(n)}=1\not=0$, 所以$f(n)\not=o(g(n))$.
	\end{enumerate}
	\jumpLine
\item 证明$lg(n!)=\Theta(nlg(n))$(课本3.19式), 并证明$n!=\omega(2^n)$且$n!=o(n^n)$
	\begin{itemize}
	\item \textbf{证明$lg(n!)=\Theta(nlg(n))$.}\\
		借助斯特林公式$$n!=\sqrt{2\pi n}(\frac{n}{e})^n(1+\Theta(\frac{1}{n}))$$
		可以得到对于充分大的n有
		$$\sqrt{2\pi n}(\frac{n}{e})^n(1+c_1\frac{1}{n})\le n!\le \sqrt{2\pi n}(\frac{n}{e})^n(1+c_2\frac{1}{n})$$
		只考虑左边的不等式, 取对数得到
		\begin{align*}
		lg(n!)&\ge nlg(\frac{n}{e})+lg(\sqrt{2\pi n}(1+c_1\frac{1}{n}))\\
		(\mbox{第二项显然大于$0$})&\ge nlg(\frac{n}{e})\\
		(\mbox{n充分大时(下面有解释)})&\ge \frac{1}{e}nlg(n)
		\end{align*}
		这里最后一步很好证明, 两边除以$n$后, 稍作移项就能得到$lg(n)-1\ge\frac{1}{e}lg(n)$, 这对于充分大的$n$是显然成立的.\\
		而又显然有
		$$lg(n!)=\sum\limits_{i=1}^nlg(i)\le nlg(n)$$
		因此对充分大的$n$, 我们有
		$$\frac{1}{e}nlg(n)\le lg(n!)\le nlg(n)$$
		因此可以说明$lg(n!)=\Theta(nlg(n))$
	\item \textbf{证明$n!=\omega(2^n)$且$n!=o(n^n)$}\\
		根据刚才得到的不等式, 即对充分大的$n$有
		$$\frac{1}{e}nlg(n)\le lg(n!)\le nlg(n)$$
		而对$n$充分大的时候,其不等号严格成立是显然的, 即
		$$\frac{1}{e}nlg(n)< lg(n!)< nlg(n)$$
		稍作变换得到
		$$lg((n^{\frac{1}{e}})^n)< lg(n!)< lg(n^n)$$
		当$n$充分大, 显然有$2<n^{\frac{1}{e}}$, 因此
		$$0\le2^n<n!<n^n$$
		故$n!=\omega(2^n)$且$n!=o(n^n)$, 证毕!
	\end{itemize}
	\jumpLine
\item 使用代入法证明$T(n)=T(\lceil n/2\rceil)+1$的解为$O(lgn)$
	\begin{itemize}
	\item 首先对于$n=1$的情况, 等式是不成立的, 不必考虑
	\item 对于初始情况n=2, 有$0\le T(2)=T(1)+1\le clg(n)$, 这里只需要c充分大即可, 因此初始情况成立
	\item 假设当$2\le k<n$时均有$T(k)=T(\lceil k/2\rceil)+1$, 那么
		\begin{align*}
		T(n)&=T(\lceil n/2\rceil)+1\\
		&\le clg(\lceil n/2\rceil)+1\\
		&\le clg(n) + clg\left(\frac{\lceil n/2\rceil}{n}\right)+1
		\end{align*}
		这里$\lceil n/2\rceil\in[\frac{1}{2},\frac{1}{2}+\frac{1}{2n})$, 因此对于$n\ge 2$, 必然有$lg\left(\frac{\lceil n/2\rceil}{n}\right)\le 0$, 这意味着对于充分大的$c$可以有
		$$T(n)\le clg(n) + clg\left(\frac{\lceil n/2\rceil}{n}\right)+1\le clg(n)$$
		因此当$k=n$时也成立
	\end{itemize}
	综合上述几点通过代入法就知: $T(n)=T(\lceil n/2\rceil)+1$的解为$O(lgn)$
	\jumpLine
\item 对递归式$T(n)=T(n-a)+T(a)+cn$, 利用递归树给出一个渐进紧确解, 其中$a\ge 1$和$c>0$为常数.\\
	\begin{forest}
	[$cn$ 
		[$T(a)$] {
			\draw[->, line width=2, dotted] (.south west)--++(-0.5,-0.5)--++(-0.5,0) node[anchor=east]{$cn+T(a)$};
		}
		[$c(n-a)$=
			[$T(a)$] {
				\draw[->, line width=2, dotted] (.south west)--++(-0.5,-0.5)--++(-0.5,0) node[anchor=east]{$c(n-a)+T(a)$};
			}
			[\dots
				[,phantom]
				[$c(n-\lfloor\frac{n}{a}\rfloor a)$
					[$T(a)$] {
						\draw[->, line width=2, dotted] (.south west)--++(-0.5,-0.5)--++(-0.5,0) node[anchor=east]{$c(n-\lfloor\frac{n}{a}\rfloor a)+T(a)$};
					}
					[$T(\frac{n}{a}-\lfloor\frac{n}{a}\rfloor)$,name=object
	]]]]];
	\end{forest}\\
	从图中可以知道(如果令n为a的倍数, 先去估计一个解)\begin{align*}
	T(n)&=\Theta(\floor{\frac{n}{a}}(cn+T(a))-\sum\limits_{i=0}^{\floor{\frac{n}{a}}}ia+T(\frac{n}{a}-\floor{\frac{n}{a}}))\\
	&=\Theta(\floor{\frac{n}{a}}(cn+T(a))-\frac{1}{2}(\floor{\frac{n}{a}})(\floor{\frac{n}{a}}+1)a+T(\frac{n}{a}-\floor{\frac{n}{a}}))\\
	&=\Theta(\frac{n}{a}(cn+T(a))-\frac{1}{2}(\frac{n}{a})(\frac{n}{a}+1)a+T(\frac{n}{a}-\frac{n}{a}))\\
	&=\Theta(\frac{n^2}{a}((c-\frac{1}{2}))+(\frac{T(a)}{a}-\frac{1}{2})n)\\
	&=\Theta(n^2)
	\end{align*}
	下面用归纳法证明$T(n)=\Theta(n^2)$.
	\begin{itemize}
	\item 假设当$k<n$的时候都有$T(n)\in[c_1k^2,c_2k^2]$, 那么考虑$k=n$的情况有:
		\begin{align*}
		T(n)=T(n-a)+T(a)+cn&\in[c_1(n-a)^2+a^2+cn,c_2(n-a)^2+a^2+cn]\\
		&=[c_1n^2+c_1(a^2-2an)+a^2+cn,c_2n^2+c_2(a^2-2an)+a^2+cn]\\
		&\subset[c_1n^2,c_2n^2](\mbox{这里取$c_1(a^2-2an)+a^2+cn\ge 0$, $c_2(a^2-2an)+a^2+cn\le0$})
		\end{align*}
		以上当$c_1\le \frac{a^2+cn}{2an-a^2}\le c_2$(其中$n\ge\frac{a}{2}$)时成立.
	\item 再考虑初始情况: 对于合适的$n_0$, 当$n<n_0$时必然有$T(n)=\Theta(1)$(这是因为在$n_0$给定的情况下, $T(n)$必然都是有限的数), 因此只需要找到足够小的$c_1$和足够大的$c_2$就能够满足$c_1n^2<T(n)<c_2n^2$ 
	\end{itemize}
	至此归纳证明结束.
	
\item 对下列递归式, 使用主方法求出渐进紧确解:
	\begin{enumerate}[(a). ]
	\item $T(n)=2T(n/4)+\sqrt{n}$
	\item $T(n)=2T(n/4)+n^2$
	\end{enumerate}
	\begin{enumerate}[(a). ]
	\item 这里a=2, b=4, 从而$n^{log_ba}=n^{log_42}=\sqrt{n}$, 显然有$\sqrt{n}=\Theta(\sqrt{n})=\Theta(n^{log_ba})$(自反性)\\
		从而可以使用主定理的第二种情况直接得出$$T(n)=\Theta(\sqrt{n}lg(n))$$
	\item 这里这里a=2, b=4, 从而$n^{log_ba+\epsilon}=n^{log_42+\epsilon}=n^{\frac{1}{2}+\epsilon}$,\\
		因此对于$\epsilon=\frac{3}{2}$有, $f(n)=n^2=\Omega(n^2)=\Omega(n^{log_42+\epsilon})$(自反性).\\
		又有, $af(n/b)=2f(n/4)=n^2/8\le\frac{1}{8}n^2$, \\
		即对充分大的$n$, 取$c=\frac{1}{8}$, 有$af(n/b)\le cf(n)$,\\
		故由主定理的第三种情况直接得到, $$T(n)=\Theta(n^2)$$
	\end{enumerate}
\item 主方法能应用于递归式$T(n)=4T(n/2)+n^2\lg(n)$吗? 请说明为什么可以或为什么不可以. 给出这个递归式的一个渐进上界.
	\begin{itemize}
	\item 这里$a=4$, $b=2$, $n^{log_ba}=n^2$. 分别考虑主方法的三种情况:
		\begin{enumerate}[(1). ]
		\item $n^{log_ba-\epsilon}=n^{2-\epsilon}$, \\若要能应用第一种情况, 则要存在$c$, 使得对充分大的$n$都有$f(n)=n^2\lg n\le cn^{2-\epsilon}$,\\
			即$c\ge n^\epsilon \lg n$, 这显然无法做到.
		\item 
			\begin{enumerate}
			\item 如果按老师的PPT: 显然有$n^2\lg n=\Theta(n^2\lg n)$, 因此可以应用第二种情况, 直接能够得到
			$$T(n)=\Theta(n^2\lg^2n)$$
			\item 如果按课本: 那么$n^2\lg n!=\Theta(n^2)$, 因此不能使用.
			\end{enumerate}
		\item 若要能应用第三种情况, 则要存在$c$, 使得对充分大的$n$都有$f(n)=n^2\lg n\ge cn^{2+\epsilon}$,\\
			即$c\le n^{-\epsilon} \lg n$, 这也是做不到的, 因为$\epsilon>0$, 不等式右边是指数衰减的(这里直接描述了, 从数学上也是很好证明的).
		\end{enumerate}
		综上所述, 如果按老师的PPT来, 就可以; 如果按课本来, 就不行. 原因上面已经说清楚了.
	\item 
		\begin{enumerate}[i.]
		\item 如果按老师的PPT: 因为刚才已经给出了渐进紧确解, 那么它也是一个渐进上界, 即$$T(n)=O(n^2\lg^2n)$$
		\item 如果按课本: 那么需要手动构造一个渐进上界. 我们可以猜测
		$$T(n)=O(n^2\lg^2n)$$
		\begin{itemize}
		\item 假设对于$k<n$, 都有$T(k)<ck^2\lg^2k$成立. 那么就有
		\begin{align*}
		T(n)&=4T(n/2)+n^2\lg n\\
		&\le 4c\left(\frac{n}{2}\right)^2\lg^2\frac{n}{2}+n^2\lg n\\
		&=cn^2\lg^2n-cn^2\lg 2+n^2\lg n\\
		&\le cn^2\lg^n-n^2(\lg n-c\lg 2)\\
		&\le cn^2\lg^2\quad(\mbox{当$\lg n>c\lg 2$即$n>2^c$})
		\end{align*}
		\item 再考虑初始情况: 对于合适的$n_0$, 当$n<n_0$时必然有$T(n)=\Theta(1)$(这是因为在$n_0$给定的情况下, $T(n)$必然都是有限的数), 因此只需要找到足够大的$c$就能够满足$T(n)<cn^2lg^n$
		\end{itemize}
		\end{enumerate}
	
	\end{itemize}
\end{enumerate}
\end{document}









