\documentclass[UTF8]{article}
\usepackage{graphicx}
\usepackage{subfigure}
\usepackage{amsmath}
\usepackage{makecell}
\usepackage[utf8]{inputenc}
\usepackage[space]{ctex} %中文包
\usepackage{listings} %放代码
\usepackage{xcolor} %代码着色宏包
\usepackage{CJK} %显示中文宏包
\usepackage{float}
\usepackage{diagbox}
\usepackage{bm}
\usepackage{ulem} 
\usepackage{amssymb}
\usepackage{soul}
\usepackage{color}
\usepackage{geometry}
\usepackage{fancybox} %花里胡哨的盒子
\usepackage{xhfill} %填充包, 可画分割线 https://www.latexstudio.net/archives/8245
\usepackage{multicol} %多栏包
\usepackage{enumerate} %可以方便地自定义枚举标题
\usepackage{multirow} %表格中多行单元格合并
\usepackage{wasysym} %可以使用wasysym里的一堆奇奇怪怪的符号
\usepackage{hyperref} % url
%%%%%%%%%%%%%%%伪代码%%%%%%%%%%%%%%%
\usepackage{amsmath}
\usepackage{algorithm}
\usepackage{algorithmicx}
\usepackage[noend]{algpseudocode}
%%%%%%%%%%%%%%%画图包%%%%%%%%%%%%%%%
\usepackage{tikz}
\usepackage{pgfplots} % http://pgfplots.sourceforge.net/gallery.html
\usetikzlibrary{pgfplots.patchplots} % 拟合支持
\usetikzlibrary{arrows,shapes,automata,petri,positioning,calc} % 状态图支持
\usetikzlibrary{arrows.meta} % 箭头
\usetikzlibrary{shadows} % 阴影支持
\usepackage{forest} % 画树

\geometry{left = 1.5cm, right = 1.5cm, top=1.5cm, bottom=2cm}

\definecolor{mygreen}{rgb}{0,0.6,0}
\definecolor{mygray}{rgb}{0.5,0.5,0.5}
\definecolor{mymauve}{rgb}{0.58,0,0.82}
\lstset{
	backgroundcolor=\color{white}, 
	%\tiny < \scriptsize < \footnotesize < \small < \normalsize < \large < \Large < \LARGE < \huge < \Huge
	basicstyle = \footnotesize,       
	breakatwhitespace = false,        
	breaklines = true,                 
	captionpos = b,                    
	commentstyle = \color{mygreen}\bfseries,
	extendedchars = false,
	frame = shadowbox, 
	framerule=0.5pt,
	keepspaces=true,
	keywordstyle=\color{blue}\bfseries, % keyword style
	language = C++,                     % the language of code
	otherkeywords={string}, 
	numbers=left, 
	numbersep=5pt,
	numberstyle=\tiny\color{mygray},
	rulecolor=\color{black},         
	showspaces=false,  
	showstringspaces=false, 
	showtabs=false,    
	stepnumber=1,         
	stringstyle=\color{mymauve},        % string literal style
	tabsize=4,          
	title=\lstname           
}

%\sum\nolimits_{j=1}^{M}   上下标位于求和符号的水平右端,
%\sum\limits_{j=1}^{M}   上下标位于求和符号的上下处,
%\sum_{j=1}^{M}  对上下标位置没有设定,会随公式所处环境自动调整。

%%%%%%%%%%%%%画图包%%%%%%%%%%%%%
\usepackage{tikz}
%%%%%%%%%%%%%好看的矩形%%%%%%%%%%%%%
\tikzset{
	rect1/.style = {
		shape = rectangle,% 指定样式
		minimum height=2cm,% 最小高度
		minimum width=4cm,% 最小宽度
		align = center,% 文字居中
		drop shadow,% 阴影
	}
}
%%%%%%%%%%%%%画图背景包%%%%%%%%%%%%%
\usetikzlibrary{backgrounds}

%%%%%%%%%%%%%在tikz中画一个顶点%%%%%%%%%%%%%
%%%%%%%%%%%%%#1:node名称%%%%%%%%%%%%%
%%%%%%%%%%%%%#2:位置%%%%%%%%%%%%%
%%%%%%%%%%%%%#3:标签%%%%%%%%%%%%%
\newcommand{\newVertex}[3]{\node[circle, draw=black, line width=1pt, scale=0.8] (#1) at #2{#3}}
%%%%%%%%%%%%%在tikz中画一条边%%%%%%%%%%%%%
\newcommand{\newEdge}[2]{\draw [black,very thick](#1)--(#2)}
%%%%%%%%%%%%%在tikz中放一个标签%%%%%%%%%%%%%
%%%%%%%%%%%%%#1:名称%%%%%%%%%%%%%
%%%%%%%%%%%%%#2:位置%%%%%%%%%%%%%
%%%%%%%%%%%%%#3:标签内容%%%%%%%%%%%%%
\newcommand{\newLabel}[3]{\node[line width=1pt] (#1) at #2{#3}}

%%%%%%%%%%%%%强制跳过一行%%%%%%%%%%%%%
\newcommand{\jumpLine} {\hspace*{\fill} \par}
%%%%%%%%%%%%%关键点指令,可用itemise替代%%%%%%%%%%%%%
\newcommand{\keypoint}[2]{$\bullet$\textbf{#1}\quad#2\par}
%%%%%%%%%%%%%<T>平均值表示%%%%%%%%%%%%%
\newcommand{\average}[1]{\left\langle #1\right\rangle }
%%%%%%%%%%%%%表格内嵌套表格%%%%%%%%%%%%%
\newcommand{\tabincell}[2]{\begin{tabular}{@{}#1@{}}#2\end{tabular}}
%%%%%%%%%%%%%大黑点item头%%%%%%%%%%%%%
\newcommand{\itemblt}{\item[$\bullet$]}
%%%%%%%%%%%%%大圈item头%%%%%%%%%%%%%
\newcommand{\itemc}{\item[$\circ$]}
%%%%%%%%%%%%%大星星item头%%%%%%%%%%%%%
\newcommand{\itembs}{\item[$\bigstar$]}
%%%%%%%%%%%%%右▷item头%%%%%%%%%%%%%
\newcommand{\itemrhd}{\item[$\rhd$]}
%%%%%%%%%%%%%定义为%%%%%%%%%%%%%
\newcommand{\defas}{=_{df}}
%%%%%%%%%%%%%偏导%%%%%%%%%%%%%
\newcommand{\partialx}[2]{\frac{\partial #1}{\partial #2}}
%%%%%%%%%%%%%蕴含%%%%%%%%%%%%%
\newcommand{\imp}{\rightarrow}
%%%%%%%%%%%%%上取整%%%%%%%%%%%%%
\newcommand{\ceil}[1]{\lceil#1\rceil}
%%%%%%%%%%%%%下取整%%%%%%%%%%%%%
\newcommand{\floor}[1]{\lfloor#1\rfloor}

%%%%%%%%%%%%%双线分割线%%%%%%%%%%%%%
\newcommand*{\doublerule}{\hrule width \hsize height 1pt \kern 0.5mm \hrule width \hsize height 2pt}
%%%%%%%%%%%%%双线中间可加东西的分割线%%%%%%%%%%%%%
\newcommand\doublerulefill{\leavevmode\leaders\vbox{\hrule width .1pt\kern1pt\hrule}\hfill\kern0pt }
%%%%%%%%%%%%%左大括号%%%%%%%%%%%%%
\newcommand{\leftbig}[1]{\left\{\begin{array}{l}#1\end{array}\right.}
%%%%%%%%%%%%%矩阵%%%%%%%%%%%%%
\newcommand{\mat}[2]{\left[\begin{array}{#1}#2\end{array}\right]}
%%%%%%%%%%%%%可换行圆角文本框%%%%%%%%%%%%%
\newcommand{\ovalboxn}[1]{\ovalbox{\tabincell{l}{#1}}}
%%%%%%%%%%%%%设置section的counter, 使从1开始%%%%%%%%%%%%%
\setcounter{section}{0}

%%%%%%%%%%%%%Colors%%%%%%%%%%%%%
\newcommand{\lightercolor}[3]{% Reference Color, Percentage, New Color Name
	\colorlet{#3}{#1!#2!white}
}
\newcommand{\darkercolor}[3]{% Reference Color, Percentage, New Color Name
	\colorlet{#3}{#1!#2!black}
}
\definecolor{aquamarine}{rgb}{0.5, 1.0, 0.83}
\definecolor{Seashell}{RGB}{255, 245, 238} %背景色浅一点的
\definecolor{Firebrick4}{RGB}{255, 0, 0}%文字颜色红一点的
\lightercolor{gray}{20}{lgray}
\newcommand{\hlg}[1]{
	\begingroup
	\sethlcolor{lgray}%背景色
	\textcolor{black}{\hl{\mbox{#1}}}%textcolor里面对应文字颜色
	\endgroup
}
\lightercolor{red}{20}{lred}
\lightercolor{black}{20}{lblack}
\lightercolor{black}{60}{mblack}

\title{算法基础 HW5}
\author{PB18111697 王章瀚}

\begin{document}
\maketitle
\section{}
\noindent \textbf{使用链表表示和加权合并启发式策略,写出{\sc Make-Set}, {\sc Find-Set}和{\sc Union}操作的伪代码}\\
\jumpLine
\noindent 详细说明已经注释清楚, 就不再赘述了.\\
\begin{minipage}{\linewidth*5/7}
	\begin{algorithm}[H]
		\caption{链表表示和加权合并启发式策略的不相交集——{\sc Make-Set}}
		\begin{algorithmic}[1] %每行显示行号
		\Function {\sc Make-Set}{int key}
			\State node = new Node
			\State disjset = new DisjSet
			\State node.key = key
			\State node.set = list \Comment{该结点的所属集合指向该链表}
			\State disjset.head = node
			\State disjset.tail = node \Comment{链表头尾指向该结点}
			\State disjset.size = 1
			\State \Return{disjset}
		\EndFunction
		\end{algorithmic}
	\end{algorithm}
		
	\begin{algorithm}[H]
		\caption{链表表示和加权合并启发式策略的不相交集——{\sc Find-Set}}
		\begin{algorithmic}[1] %每行显示行号
		\Function {\sc Find-Set}{x}
			\State \Return{x.set} \Comment{直接返回其对应集合}
		\EndFunction
		\end{algorithmic}
	\end{algorithm}
		
	\begin{algorithm}[H]
		\caption{链表表示和加权合并启发式策略的不相交集——{\sc Union}}
		\begin{algorithmic}[1] %每行显示行号
		\Function {\sc Union}{disjset1, disjset2}
			\If {disjset1.size < disjset2.size}
				\State {\sc Swap}(disjset1, disjset2) \Comment{启发式: 保证disjset1大小比较小}
			\EndIf
			\For{node in disjset2} \Comment{将disjset2的结点的集合改为disjset1}
				\State node.set = disjset1
			\EndFor
			\State disjset1.tail.next = disjset2.head
			\State disjset1.tail = disjset2.tail \Comment{将disjset2的结点加入disjset1}
			\State disjset2.head = zdisjset2.tail = NULL \Comment{清空disjset2}
			\State disjset1.size = disjset1.size + disjset2.size \Comment{更新disjset1大小}
			\State disjset2.size = 0 \Comment{更新disjset2大小}
		\EndFunction
		\end{algorithmic}
	\end{algorithm}
\end{minipage}

\newpage
\section{}
\textbf{设定动态规划算法求解0-1背包问题,要求运行时间为O(nW),n为商品数量,W是小偷能放进背包的最大商品总重量}\\
\begin{minipage}{\linewidth*6/7}
	\begin{algorithm}[H]
		\caption{0-1背包问题}
		\begin{algorithmic}[1] %每行显示行号
		\Require weight: 重量数组, value: 价值数组, n: 数目, W: 背包最大重
		\Function {\sc Package01}{weight, value, n, W}
			\State r = new array[W] \Comment{r[i]表示大小为r[i]的包最大价值}
			\For{i = 1 $\to$ W} \Comment{放第0个物品}
				\State r[i] = weight[0] $\le$ i ? value[0] : 0 \Comment{当放得下物品i的时候才放}
			\EndFor
			\For{i = 1 $\to$ n-1} \Comment{考虑1-n的每个物品是否放}
				\For{j = W $\to$ weight[i]} \Comment{考虑所有能放下物品i的包大小}
					\State r[j] = $\max$(r[j], r[j-weight[i]]+value[i]) \Comment{当取了总价值能变大才取}
				\EndFor
			\EndFor
			\Return r[W] \Comment{返回最大总价值}
		\EndFunction
		\end{algorithmic}
	\end{algorithm}
\end{minipage}


\end{document}









